\documentclass{tccv}
\usepackage[english]{babel}

\begin{document}

\part{William Seaford}

\section{Skill Set}

\begin{factlist}

\item{Languages}
     {Java, C, Python}
     
\item{Specialist Java}
	 {Hadoop (Map/Reduce), EJB, J2EE, RESTful API}
	 
\item{Databases}
	 {SQL}

\item{Operating Systems}
	{\underline{UNIX/Linux}\newline RHEL, centOS, Fedora, Ubuntu, Mint  \newline \underline{Windows}\newline XP, Vista, 7, 8/8.1, 10}
\end{factlist}

\section{Work experience}

\begin{eventlist}

\item{March 2016 -- Present}
     {IT Consultant}
     {QA Consulting, Manchester}

Contracted to Capgemini for research \& development of a cloud based platform using tools such as the Amazon web services’ EC2 and Hashicorp’s Vagrant and Terraform to provision environments for use in the context of API provisioning, identity management, automated provisioning and service management.
\newline
\newline
Strong use of UNIX based operating systems as well as OS X. Worked within an agile scrum environment, which included daily standups, sprint reviews and retrospectives. My contributions include implementing an automated service desk, researching advanced security software such as OSSec \& SELinux as well as provisioning environments for pre-release clients. 

\item{January 2015 -- April 2015}
	 {Teaching Assistant}
	 {Brampton Manor Academy, East Ham}
	 
Assigned a 10 week placement for a university module working with a year 8 mixed ability class. Duties included assisting the teacher in aiding learning to pupils and answering any technical questions students had relating towards syllabus and extracurricular topics relating towards technology.  
\newline
\newline
Given control of the classroom for a period of time each week to teach material derived from own lesson plan with suitable exercises tasked for students afterwards to inspire further learning. Highly positive feedback given and
thorough enjoyment expressed from all students. Demonstrated excellent aptitude for teaching.

\end{eventlist}

\personal
    [www.github.com/willseaford]
    {Wellington, Shropshire\newline United Kingdom, TF1 2ED}
    {+44 (0) 7561 358 418}
    {willseaford[AT]gmail[DOT]com}
    {www.willseaford.co.uk}

\section{Education}

\begin{yearlist}

\item[2:1 BSc Computer\newline Science(Hons)]{2012 -- 2015}
     {Queen Mary, University of London}

\item[\underline{A/Level:}\newline Economics, Psychology\newline Mathematics]{2010 -- 2012}
     {Richard Huish College}
     {\underline{AS/Level:}\newline Business Studies, Computing\newline Extended Project Qualification}

\end{yearlist}

\section{Dissertation}

Investigated the effects of bottlenecking within a butterfly network graph. Took an in depth look at
the research paper "network routing capacity" by J.Cannons and extracted the algorithm proposed,
applying it towards the network graph and examined the effects certain edges with limited capacity
has on the routing capacity. Topic included a review of network coding (coding information as a
function at the node) and featured strong use of linear programming within Mathematica. Grade
A achieved with comments made by examiners praising use of critical thinking and approach to a
technically hard subject.

\section{Past Projects}

Group projects include creating a Hospital/GP application which made use of a database through
Java Database Connectivity (JDBC). Role was to create the  staff records  module and assist in
integrating it with other member's modules. Was a team leader in building a three screened mobile
weather application with specific design towards early morning alarm and forecasting. Completed
team data mining tasks set on Kaggle.com such as the titanic task where predictions of variables
were made using existing data using regression.
\newline \newline
Individual projects include creating a simple fantasy game which demonstrated various programming concepts, flash card application to aid academic revision, film quiz with various difficulties and question sets. Also has high interest in solving problems posed on websites such Project-Euler and various forum boards with notable solutions on Github.

\end{document}